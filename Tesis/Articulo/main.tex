\documentclass[aps,prd,nofootinbib,superscriptaddress,floatfix,longbibliography,author-year]{revtex4-2}

% Basic packages
\usepackage[utf8]{inputenc}
\usepackage[T1]{fontenc}
\usepackage{lmodern}                  % Modern, clear font
\usepackage{amsmath,amssymb,mathtools}% Math support
\usepackage{graphicx}                 % For figures
\usepackage{hyperref}                % Hyperlinks
\usepackage[english]{babel}
\usepackage{color}                   % Color support
\usepackage{booktabs}                % For better tables
\usepackage{multirow}                % For multirow tables
\usepackage{array}                   % For better table column definitions
\usepackage{dcolumn}                 % For decimal point alignment in tables
\usepackage{float}                   % For figure placement
\usepackage{placeins}                % For figure placement control
%\usepackage[authoryear]{natbib}


\hypersetup{
    colorlinks=true,
    linkcolor=blue,
    citecolor=blue,
    urlcolor=blue
}

% Units and physical constants
\usepackage{siunitx}
\sisetup{detect-all}

% Custom commands (optional)
\newcommand{\dif}{\mathrm{d}}

% Bibliography
\bibliographystyle{apsrev4-2}



\begin{document}

\title{Multispectral and Morphological Analysis of Garavito Craters in the South Pole - Aitken Basin on the Moon}

\author{Eduardo A. Delgadillo M}
\author{Mario A. Higuera}
\affiliation{Observatorio Astronómico Nacional, Universidad Nacional de Colombia, Bogotá, Colombia}

\author{David Ardila R.}
\affiliation{Jet Propulsion Laboratory, California Institute of Technology, Pasadena, CA 91109, USA}

\date{\today}

\begin{abstract}
Garavito craters form a large, complex area within the South Pole–Aitken basin on the far side of the Moon. Here, we present the results of a detailed multispectral and morphological analysis of the Garavito region, using datasets from several missions. Our findings indicate that this complex area exhibits diverse geological origins, which have determined its environmental conditions.
We classified the five Garavito craters based on depth, slope, shape, and estimated formation. Additionally, we determined their sizes using elevation profiles and conducted a spectral analysis with M3 data to identify the minerals present on the surface. Spectral analyses indicate that the Garavito region is predominantly covered by pyroxene. Furthermore, by applying the thermal removal methodology developed by Clark, we determined the temperatures in this area.
\end{abstract}

\maketitle

\section{Introduction}
The evolution of the Lunar environment has been controlled by different factors, the most important are volcanism, tectonics, impacts and exposure to interplanetary space. These processes have shaped the Lunar surface along its history, over 4 billion years. Topography on the lunar near side is divided almost equally between flat basaltic maria and feldspathic highlands, whereas the far side is dominated by highlands crust (\cite{JAUMANN201215}).  Currently, the study and understanding of the Moon rely primarily on remote sensing observations and the in situ collection of samples for laboratory analysis. In this context, two key factors shape our current knowledge of the lunar surface: the widespread presence of impact craters, which define much of the Moon’s topography, and the existence of a thin, heterogeneous layer that covers the entire surface—the regolith. This mantle consists of fragmental, unconsolidated rock material, either residual or transported, and exhibits a highly variable composition depending on location and geologic context.\\

The most important crater structure over the moon is the the South Pole–Aitken basin, which is one of the largest impact basins in the solar system, and it is believed to have formed during a period of intense bombardment in the early history of the Moon (\citet{JAUMANN201215}). The Von Kármán crater formed in the Nectarian epoch with its floor flooded by low-albedo mare basalt in the late Imbrian
age (\citet{PASCKERT2018538}) This event was a several influence in hole lunar estructure evolution,for example the elevation range on the moon, is a overall above a reference perfect sphere of 1737.4 km radius, with a minimum from about -9100 $m$, in the South Pole-Aitken basin, up to about 10770 $m$ in the far side highlands and the fact that the lunar nearside is dominated by mare volcanism and the farside shows only isolated mare deposits within large craters and basins (\citet{JAUMANN201215}).In the highlands,the dominant minerals are calcium plagioclases (\citet{TAYLOR1972263}) while in the maria areas compositions shows higher abundances of clinopyroxene (CPX), orthopyroxene (OPX), and olivine (\citet{ALBEE2003825}).The local topography may strongly affect the surface temperature and thus control the retention of surface water (\citet{LI2025116668}).\\

The Garavito craters is a large region, composed for five craters in  the far side of the moon, which is located around of $47.285^\circ$ $S$, $157.137^\circ$ $E$ inside of the South Pole–Aitken basin. Julio Garavito Armero was a famous Colombian astronomer, engineer and mathematician, who was born in 1865 and died in 1920. He is known for his work on the orbits of asteroids, as well as for his contributions to the field of celestial mechanics (\citet{book:Brouwer1961}), focused in the movement of the Moon theory(\citet{Sanchez2025}). The first of the Garavito craters was named in 1970 by the International Astronomical Union (IAU) in honor of the Astronomer. The other four craters were named in 1973, 1985, 1994 and 1997 respectively. The Garavito craters are located in a region of the Moon that is thought to be one of the oldest and most geologically complex areas on the lunar surface.

Areas marked as mare are dark and very flat, as expected for “exposed” basaltic lava flows (\citet{Gibson2011}).
\section{Theoretical Background}
This section describes the theoretical basis and relevant equations.

\section{Methodology}
This section explains the procedure, data, tools, and software used.

\section{Results and Analysis}
This section presents the obtained results and corresponding analysis. You can include figures like:
\begin{figure}[h]
    \centering
    \includegraphics[width=0.45\textwidth]{Images/Ellipse_region.png}
    \caption{Description of the figure.}
    \label{fig:example}
\end{figure}

\section{Conclusions}
Discuss the main findings and potential future extensions of the work.

\section*{Acknowledgments}
Optional: mention people or institutions that supported the project.

\bibliography{ref}  % references.bib file

\end{document}
